\documentclass{article}
\usepackage{graphicx}
\usepackage{fancyhdr}
\usepackage{listings}

\let\<\textless
\let\>\textgreater

\graphicspath{ {images/} }
\pagestyle{fancy}
\fancyhf{}
\rhead{Proyecto \#2}
\rfoot{P\'agina \thepage}

\begin{document}
\begin{titlepage}
  \centering
  {\scshape\LARGE Instituto Tecnol\'ogico de Costa Rica \par}
  \vspace{1cm}
  {\scshape\Large Redes\par}
  {\scshape\Large Proyecto \#1 - WaveNET\par}
  \vspace{1.5cm}
  {\Large\itshape Allan Rojas\par}
  {\Large\itshape Sa\'ul Zamora\par}
  \vfill
  profesor\par
  Kevin Moraga \textsc{}

  \vfill

% Bottom of the page
  % {\large \today\par}
\end{titlepage}

\section{Introducci\'on}
La comunicaci\'on es un proceso fundamental para compartir ideas. Sin embargo, la brecha tecnol\'ogica y de comunicaci\'on es m\'as notoria en \'areas donde los insumos que poseen las TELCOS para implementar las soluciones necesarias, son bajos.
Debido a esto, es una buena opci\'on producir nuevos m\'etodos alternativos de comunicaci\'on. Un ejemplo son las redes mesh de proyectos como guifi.net y openmesh.
Dado lo anterior, el objetivo del presente proyecto es conocer distintas soluciones que nos permitan acercarnos m\'as a esa libertad de expresi\'on y disminuir la brecha tecnol\'ogica.

\section{Ambiente de desarrollo}
\begin{itemize}
  \item Raspberry Pi:
  \begin{itemize}
    \item Python
  \end{itemize}
  \item Receptor:
  \begin{itemize}
    \item Java
  \end{itemize}
\end{itemize}

\section{Estructuras de datos usadas y funciones}
\section{Instrucciones de ejecuci\'on}
\section{Bit\'acora de trabajo}
\section{Comentarios finales}
\section{Conclusiones}

\begin{thebibliography}{9}
  \bibitem{Gpiozero}
  Gpiozero.readthedocs.io. (2018). \emph{13. API - Output Devices — Gpiozero 1.4.1 Documentation.}
  [online] Available at: \url{https://gpiozero.readthedocs.io/en/stable/api\_output.html\#buzzer}
  
  \bibitem{Scapy}
  community., P. (2018). \emph{Scapy.}
  [online] Scapy.net. Available at: \url{https://scapy.net/}
  
  \bibitem{Scapygit} GitHub. (2018). \emph{secdev/scapy.}
  [online] Available at: \url{https://github.com/secdev/scapy}
\end{thebibliography}

\end{document}
