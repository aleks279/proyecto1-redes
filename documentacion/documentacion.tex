\documentclass{article}
\usepackage{graphicx}
\usepackage{fancyhdr}
\usepackage{listings}

\let\<\textless
\let\>\textgreater

\graphicspath{ {images/} }
\pagestyle{fancy}
\fancyhf{}
\rhead{Proyecto \#2}
\rfoot{P\'agina \thepage}

\begin{document}
\begin{titlepage}
  \centering
  {\scshape\LARGE Instituto Tecnol\'ogico de Costa Rica \par}
  \vspace{1cm}
  {\scshape\Large Redes\par}
  {\scshape\Large Proyecto \#1 - WaveNET\par}
  \vspace{1.5cm}
  {\Large\itshape Allan Rojas\par}
  {\Large\itshape Sa\'ul Zamora\par}
  \vfill
  profesor\par
  Kevin Moraga \textsc{}

  \vfill

% Bottom of the page
  % {\large \today\par}
\end{titlepage}

\section{Introducci\'on}
La comunicaci\'on es un proceso fundamental para compartir ideas. Sin embargo, la brecha tecnol\'ogica y de comunicaci\'on es m\'as notoria en \'areas donde los insumos que poseen las TELCOS para implementar las soluciones necesarias, son bajos.
Debido a esto, es una buena opci\'on producir nuevos m\'etodos alternativos de comunicaci\'on. Un ejemplo son las redes mesh de proyectos como guifi.net y openmesh.
Dado lo anterior, el objetivo del presente proyecto es conocer distintas soluciones que nos permitan acercarnos m\'as a esa libertad de expresi\'on y disminuir la brecha tecnol\'ogica.

\section{Ambiente de desarrollo}
\begin{itemize}
  \item Raspberry Pi:
  \begin{itemize}
    \item Python
  \end{itemize}
  \item Receptor:
  \begin{itemize}
    \item Java
  \end{itemize}
\end{itemize}

\section{Estructuras de datos usadas y funciones}
\section{Instrucciones de ejecuci\'on}
\section{Bit\'acora de trabajo}
\subsection{Sa\'ul Zamora}
\begin{itemize}
  \item 23-09-2018:
  \begin{itemize}
    \item 2 horas - Investigar RFC.
  \end{itemize}
  \item 24-09-2018:
  \begin{itemize}
    \item 4 horas - Investigar Scapy.
  \end{itemize}
  \item 25-09-2018:
  \begin{itemize}
    \item 2 horas - Investigar scraping.
  \end{itemize}
  \item 26-09-2018:
  \begin{itemize}
    \item 2 horas - Investigar scraping.
  \end{itemize}
  \item 27-09-2018:
  \begin{itemize}
    \item 2 horas - Investigar onion routing.
  \end{itemize}
  \item 28-09-2018:
  \begin{itemize}
    \item 2 horas - Investigar onion routing.
  \end{itemize}
  \item 29-09-2018:
  \begin{itemize}
    \item 5 horas - Investigar audio format en Java.
  \end{itemize}
  \item 30-09-2018:
  \begin{itemize}
    \item 2 horas - Investigar scraping.
  \end{itemize}
  \item 01-10-2018:
  \begin{itemize}
    \item 2 horas - Investigar como hacer el relay chat.
    \item 2 horas - Investigar sobre el servidor IRC.
  \end{itemize}
  \item 02-10-2018:
  \begin{itemize}
    \item 2 horas - Documentaci\'on.
  \end{itemize}
\end{itemize}
Total de horas trabajadas: 27 horas.

\section{Comentarios finales}
\section{Conclusiones}

\begin{thebibliography}{9}
  \bibitem{Gpiozero}
  Gpiozero.readthedocs.io. (2018). \emph{13. API - Output Devices — Gpiozero 1.4.1 Documentation.}
  [online] Available at: \url{https://gpiozero.readthedocs.io/en/stable/api\_output.html\#buzzer}
  
  \bibitem{Scapy}
  community., P. (2018). \emph{Scapy.}
  [online] Scapy.net. Available at: \url{https://scapy.net/}
  
  \bibitem{Scapygit} GitHub. (2018). \emph{secdev/scapy.}
  [online] Available at: \url{https://github.com/secdev/scapy}

  \bibitem{hashlib} Docs.python.org. (2018). \emph{15.1. hashlib — Secure hashes and message digests — Python 3.3.7 documentation.}
  [online] Available at: \url{https://docs.python.org/3.3/library/hashlib.html}

  \bibitem{ieee} \emph{Ieee802.org.} (2018).
  [online] Available at: \url{http://www.ieee802.org/IEEE-802-LMSC-OverviewGuide-02SEPT%202012.pdf}

  \bibitem{scraping} scraping, T. (2018). \emph{Tor IP changing and web scraping.}
  [online] Dm295.blogspot.com. Available at: \url{https://dm295.blogspot.com/2016/02/tor-ip-changing-and-web-scraping.html}

  \bibitem{eprint} \emph{Eprint.iacr.org.} (2018).
  [online] Available at: \url{https://eprint.iacr.org/2011/308.pdf}

  \bibitem{onion} Onion-router.net. (2018). \emph{Onion Routing: Path Selection Algorithms.}
  [online] Available at: \url{https://www.onion-router.net/Archives/Route/Alg1/ThreeConnected.html}

  \bibitem{routing} Onion-router.net. (2018). \emph{Onion Routing: Investigation of Route Selection Algorithms.}
  [online] Available at: \url{https://www.onion-router.net/Archives/Route/index.html}

  \bibitem{chat} Es.wikipedia.org. (2018). \emph{Internet Relay Chat.}
  [online] Available at: \url{https://es.wikipedia.org/wiki/Internet_Relay_Chat}

  \bibitem{servidor} Es.tldp.org. (2018). \emph{Instalación y configuración de un servidor de IRC: Introducción.}
  [online] Available at: \url{http://es.tldp.org/COMO-INSFLUG/COMOs/Servidor-IRC-Como/Servidor-IRC-Como-2.html}

  \bibitem{rfc} \emph{Rfc-editor.org. (2018).}
  [online] Available at: \url{http://www.rfc-editor.org/rfc/rfc1459.txt}
\end{thebibliography}

\end{document}
