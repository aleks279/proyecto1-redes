\documentclass{article}
\usepackage{graphicx}
\usepackage{fancyhdr}
\usepackage{listings}

\let\<\textless
\let\>\textgreater

\graphicspath{ {images/} }
\pagestyle{fancy}
\fancyhf{}
\rhead{Proyecto \#2}
\rfoot{P\'agina \thepage}

\begin{document}
\begin{titlepage}
  \centering
  {\scshape\LARGE Instituto Tecnol\'ogico de Costa Rica \par}
  \vspace{1cm}
  {\scshape\Large Redes\par}
  {\scshape\Large Proyecto \#1 - WaveNET\par}
  \vspace{1.5cm}
  {\Large\itshape Allan Rojas\par}
  {\Large\itshape Sa\'ul Zamora\par}
  \vfill
  profesor\par
  Kevin Moraga \textsc{}

  \vfill

% Bottom of the page
  % {\large \today\par}
\end{titlepage}

\section{Introducci\'on}
La comunicaci\'on es un proceso fundamental para compartir ideas. Sin embargo, la brecha tecnol\'ogica y de comunicaci\'on es m\'as notoria en \'areas donde los insumos que poseen las TELCOS para implementar las soluciones necesarias, son bajos.
Debido a esto, es una buena opci\'on producir nuevos m\'etodos alternativos de comunicaci\'on. Un ejemplo son las redes mesh de proyectos como guifi.net y openmesh.
Dado lo anterior, el objetivo del presente proyecto es conocer distintas soluciones que nos permitan acercarnos m\'as a esa libertad de expresi\'on y disminuir la brecha tecnol\'ogica.

\section{Ambiente de desarrollo}
\begin{itemize}
  \item Raspberry Pi:
  \begin{itemize}
    \item Python
  \end{itemize}
  \item Receptor:
  \begin{itemize}
    \item Java
  \end{itemize}
\end{itemize}

\section{Estructuras de datos usadas y funciones}
\begin{itemize}
  \item Se hace uso de Java \emph{AudioFormat} para examinar y luego interpretar el formato de los datos de entrada.
  \item Se usa \emph{DataLine.Info} para guardar el tama\~no de buffer interno de almacenamiento.
  \item Se usa \emph{TargetDataLine} para leer el audio capturado por el buffer del \emph{DataLine}.
  \item Se usa una base de datos MySQL en el Raspberry para llevar cuenta de los clientes conectados a la red.
\end{itemize}

Con las clases e interfaces mencionadas, se hace un ciclo para leer el flujo de datos entrante, definir el formato y decifrar el contenido. Luego se imprime el mensaje recibido en la consola.

\section{Instrucciones de ejecuci\'on}
\subsection{WaveNET}
\begin{itemize}
  \item Ejecutar el archivo \emph{WaveNet.java}
  \item Luego de ejecutar el archivo, se tiene acceso a los siguientes comandos:
  \begin{itemize}
    \item \emph{/connect ( IP address )} : configura la IP. Identificiaci\'on con el dispositivo de transmisi\'on.
    \item \emph{/listen()} : Se prepara para recibir datos.
    \item \emph{/showFiles()} : Muestra una lista de archivos disponibles en la red.
    \item \emph{/Disconnect()} : Termina la conexi\'on activa.
  \end{itemize}
\end{itemize}

\subsection{Dispositivo de Transmisi\'on}
\begin{itemize}
  \item Ejecutar \emph{python chat-server.py}
  \item Luego de arrancar el servidor, se tiene acceso a los siguientes comandos:
  \begin{itemize}
    \item \emph{/showNodes} : Muestra una lista con los dispositivos conectados a WaveNet.
    \item \emph{/selectNode} : Da la lista de dispositivos conectados donde se puede seleccionar uno para el envio de datos a trav\'ez de la red mesh.
    \item \emph{/sendFile (filename, node)} : Desde el directorio principal, se selecciona un archivo para transmitir (path del archivo) y el dispositivo de destino.
    \item \emph{/Disconnect} : WaveNet no permite acceso a m\'as dispositivos a la red.
  \end{itemize}
\end{itemize}
\section{Bit\'acora de trabajo}
\subsection{Allan Rojas}
\begin{itemize}
  \item 29-09-2018:
  \begin{itemize}
    \item 3 horas – Buzzer Python
  \end{itemize}
  \item 29-09-2018:
  \begin{itemize}
    \item 4 horas – Database for Node Directory with MariaDB
  \end{itemize}
  \item 30-09-2018:
  \begin{itemize}
    \item 4 horas – Java Audio Listening Programming
  \end{itemize}
  \item 30-09-2018:
  \begin{itemize}
    \item 3 horas – Media Access to Raspberry
  \end{itemize}
  \item 01-10-2018:
  \begin{itemize}
    \item 5 horas – Mac Address Get / Add to Package
  \end{itemize}
  \item 10-10-2018:
  \begin{itemize}
    \item 6 horas – Package Generation and Onion Routing
  \end{itemize}
  \item 19-10-2018:
  \begin{itemize}
    \item 4 horas – Scapy Implementation add to Buzzer
  \end{itemize}
  \item 29-10-2018:
  \begin{itemize}
    \item 6 horas – Scapy Implementation add to Buzzer
  \end{itemize}
\end{itemize}
Total de Horas Trabajadas : 35

\subsection{Sa\'ul Zamora}
\begin{itemize}
  \item 23-09-2018:
  \begin{itemize}
    \item 2 horas - Investigar RFC.
  \end{itemize}
  \item 24-09-2018:
  \begin{itemize}
    \item 4 horas - Investigar Scapy.
  \end{itemize}
  \item 25-09-2018:
  \begin{itemize}
    \item 2 horas - Investigar scraping.
  \end{itemize}
  \item 26-09-2018:
  \begin{itemize}
    \item 2 horas - Investigar scraping.
  \end{itemize}
  \item 27-09-2018:
  \begin{itemize}
    \item 2 horas - Investigar onion routing.
  \end{itemize}
  \item 28-09-2018:
  \begin{itemize}
    \item 2 horas - Investigar onion routing.
  \end{itemize}
  \item 29-09-2018:
  \begin{itemize}
    \item 5 horas - Investigar audio format en Java.
  \end{itemize}
  \item 30-09-2018:
  \begin{itemize}
    \item 2 horas - Investigar scraping.
  \end{itemize}
  \item 01-10-2018:
  \begin{itemize}
    \item 2 horas - Investigar como hacer el relay chat.
    \item 2 horas - Investigar sobre el servidor IRC.
  \end{itemize}
  \item 02-10-2018:
  \begin{itemize}
    \item 2 horas - Documentaci\'on.
  \end{itemize}
  \item 27-10-2018:
  \begin{itemize}
    \item 4 horas - Documentaci\'on y RFC.
  \end{itemize}
  \item 29-10-2018:
  \begin{itemize}
    \item 2 horas - Documentaci\'on y RFC.
  \end{itemize}
\end{itemize}
Total de horas trabajadas: 33 horas.

\section{Comentarios finales}
\begin{itemize}
  \item El sistema maneja la conecci\'on de nuevos nodos (clientes) en una base de datos.
  \item Se maneja el control de direcciones MAC.
  \item Los clientes no distinguen si los paquetes recibidos son para ellos, entonces simplemente reciben y leen todos los paquetes, no desechan nada.
  \item El algoritmo de cebolla todavia esta incompleto.
\end{itemize}

\section{Conclusiones}
\begin{itemize}
  \item Uno como usuario nomarlmente toma por sentado el funcionamiento de las redes de Internet. Despu\'es de intentar implementar una red con un comportamiento similar al de la red de Internet, es evidente la complejitud que lleva solucionar un problema como este.
  \item A diferencia del proceso de comunicaci\'on convencional (conversaciones, dialogos, etc), las implementaciones digitales de dicho proceso plasman todas las complejidades del proceso que como seres humanos simplemente ya sabemos o tomamos por sentado.
\end{itemize}

\begin{thebibliography}{9}
  \bibitem{Gpiozero}
  Gpiozero.readthedocs.io. (2018). \emph{13. API - Output Devices — Gpiozero 1.4.1 Documentation.}
  [online] Available at: \url{https://gpiozero.readthedocs.io/en/stable/api\_output.html\#buzzer}
  
  \bibitem{Scapy}
  community., P. (2018). \emph{Scapy.}
  [online] Scapy.net. Available at: \url{https://scapy.net/}
  
  \bibitem{Scapygit} GitHub. (2018). \emph{secdev/scapy.}
  [online] Available at: \url{https://github.com/secdev/scapy}

  \bibitem{hashlib} Docs.python.org. (2018). \emph{15.1. hashlib — Secure hashes and message digests — Python 3.3.7 documentation.}
  [online] Available at: \url{https://docs.python.org/3.3/library/hashlib.html}

  \bibitem{ieee} \emph{Ieee802.org.} (2018).
  [online] Available at: \url{http://www.ieee802.org/IEEE-802-LMSC-OverviewGuide-02SEPT%202012.pdf}

  \bibitem{scraping} scraping, T. (2018). \emph{Tor IP changing and web scraping.}
  [online] Dm295.blogspot.com. Available at: \url{https://dm295.blogspot.com/2016/02/tor-ip-changing-and-web-scraping.html}

  \bibitem{eprint} \emph{Eprint.iacr.org.} (2018).
  [online] Available at: \url{https://eprint.iacr.org/2011/308.pdf}

  \bibitem{onion} Onion-router.net. (2018). \emph{Onion Routing: Path Selection Algorithms.}
  [online] Available at: \url{https://www.onion-router.net/Archives/Route/Alg1/ThreeConnected.html}

  \bibitem{routing} Onion-router.net. (2018). \emph{Onion Routing: Investigation of Route Selection Algorithms.}
  [online] Available at: \url{https://www.onion-router.net/Archives/Route/index.html}

  \bibitem{chat} Es.wikipedia.org. (2018). \emph{Internet Relay Chat.}
  [online] Available at: \url{https://es.wikipedia.org/wiki/Internet_Relay_Chat}

  \bibitem{servidor} Es.tldp.org. (2018). \emph{Instalación y configuración de un servidor de IRC: Introducción.}
  [online] Available at: \url{http://es.tldp.org/COMO-INSFLUG/COMOs/Servidor-IRC-Como/Servidor-IRC-Como-2.html}

  \bibitem{rfc} \emph{Rfc-editor.org. (2018).}
  [online] Available at: \url{http://www.rfc-editor.org/rfc/rfc1459.txt}
\end{thebibliography}

\end{document}
